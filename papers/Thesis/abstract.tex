Speech Recognition is a growing field since last five decades and there have been many advancements which has led to its applications like Speech to Text which allowed the possibility of Transcription of speech audio to text. Much of work is available on this in English, Arabic and Cantonese Languages. However, Urdu is a low-resource language in field of ASR although it is the world's $11^{th}$ most widely spoken language, with 232 Million speakers worldwide. We found no applicable models in our research to readily deploy Speech To Text in a noisy telephonic scenario. Apart from that we faced code-switching problem i.e. in normal telephonic or call-center conversations in Urdu, people tend to spontaneously use words from other language since Pakistan is a multi-cultural society. Hence, we proposed an implementation of Automatic Speech Recognition/ Speech to Text System in a noisy/ call center environment with less labelled training data available using Hybrid HMM-DNN in a Resource constraint environment in terms of time, budget, computation power, HR etc. We were able to access to call center large amount of unlabelled audio dataset, thanks to CPLC \cite{cplc_cplc_nodate} (a semi-government Law Enforcement Agency), some of which were labelled manually. We further integrated various open source data-sets to include more variety in data-set. The data comprised of mix of noisy and clean audio as well as single utterances and long sentences (1-20 second audios). It was split into 6.5 hours and 3.5 hours of train and test data-set respectively. The Language Model was developed from the training data-set and for acoustic modelling we used HMM (Monophone and Triphone) based on which we trained a Neural Network based model using Chain CNN-TDNN, achieving up to 5.2\% WER with noisy and clean data-set as well as on single word to spontaneous speech data as well.

\textbf{\\ Keywords:} \textit{\\ Speech Recognition, ASR, Call Center, audio transcription, Urdu language, Code-switched Urdu ASR, Speech to Text, AI, Cyber Security, ASR for Resource Constrained Environment, ASR for Noisy Environment, ASR for Low Resource Languages, Under Resourced Languages}